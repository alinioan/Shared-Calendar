\documentclass[a4paper,12pt]{article}

\usepackage[utf8]{inputenc}
\usepackage[romanian]{babel}
\usepackage{amsmath}
\usepackage{amssymb}
\usepackage{graphicx}
\usepackage{hyperref}
\usepackage{geometry}
\usepackage{float}
\usepackage{booktabs}
\usepackage{makecell}
\geometry{margin=1in}

\title{Platformă Distribuită pentru Organizarea Evenimentelor și Sincronizarea Calendarului de Grup}
\author{Alexandru Alin-Ioan \\ \small{Grupa: 344C3}}
\date{\today}

\begin{document}

\maketitle

\tableofcontents

\section{Introducere}

  Această aplicție web propune o soluție pentru un calendar distribuit prin care
  utilizatorii să poată crea, edita și sincroniza evenimente în cadrul unor
  grupuri. Platforma va oferi funcționalitatea de gestionare a disponibilității
  tutror participanților și de găsire automată de intrevale libere pentru
  evenimentele planificate.

\section{Obiectivele aplicației}

  \begin{itemize}
    \item Oferirea unui calendar multi-utilizator cu sincronizare în timp real.
    \item Gestionarea disponibilității și a evenimentelor unui participant sau
    unui grup.
    \item Prevenirea conflictelor de rezervare prin mecanisme de distributed
    locking.
    \item Generarea de propuneri automate de meeting folosind procesare
    asincronă și workeri replicați.
    \item Implementarea unor mecanisme robuste de autentificare, autorizare și
    management al rolurilor folosind Keycloak.
    \item Livrarea proiectului sub forma unui set de servicii Docker.
  \end{itemize}

\section{Modulele aplicației}

  \begin{enumerate}
    \item \textbf{Autentificare} -- funcție de autorizare utilizând o tehnologie
    de Single Sign-On (SSO).
    \item \textbf{Profil utilizator + roluri} -- Keycloak pentru managementul
    utilizatorilor, rolurilor și token-urilor.
    \item \textbf{Baza de date} -- gestionată prin ORM.
    \item \textbf{Conflict resolution + Distributed Locking} -- gestionează
    situațiile concurente în care mai mulți utilizatori creează sau modifică
    evenimente în același interval temporal.
    \item \textbf{Suggestion Worker replicat} -- calculează automat intervalele comune
    libere pentru un grup. Joburile sunt transmise de către serviciul Calendar
    către broker, care mai apoi le redistribuie către workeri.
  \end{enumerate}

\section{Descrierea arhitecturii}

  \subsection{Componente principale}
    \subsubsection{Serviciul de autentificare (Keycloak)}
      Folosește OAuth2 și emite token-uri JWT. Creează noi utilizatori, roluri
      și trimit aceste informații către serviciul de profil utilizatori
    \subsubsection{Serviciul de profil utilizatori}
      Preia informațile oferite de Keycloak și creează profiluri locale în baza
      de date. Se vor oferi 3 tipuri de utilizatori:
      \begin{enumerate}
        \item \textbf{Participant} -- Are acces la calendarul grupului și își poate
        seta disponibiltatea.
        \item \textbf{Organizer} -- Poate planifica evenimente în funcție de
        disponibiltatea participanților din grup, să adauge sau să elimine
        utilzatori din grup și să desemneze alți utilizatori ca organizatori.
        \item \textbf{Admin} -- Are acces la toate utilitatile administrative
        pentru alți utilizatori.
      \end{enumerate}  
    \subsubsection{Baza de date (PostgreSQL)}
      Baza de date va fi prezntă în două instanțe cu roluri diferite: Prima
      instanță va păstra informațile despre utilizatori și rolurile acestora,
      iar cea de a doua va reține informații despre calendar, intervalele
      ocpuate și evenimentele prezente.
    \subsubsection{Serviciul Calendar - replicat}
      Expune un API pentru managementul grupului, operații CRUD pentru
      evenimente și verificarea disponibilității. Serviciul Calendar este
      replicat pentru a gestiona un număr mare de cereri simultane. Deoarece
      logica sa este stateless, iar mecanismele de sincronizare sunt gestionate
      extern prin Redis (locking) și RabbitMQ (comunicare asincronă),
      replicarea nu introduce probleme de consistență și îmbunătățește
      disponibilitatea sistemului.
    \subsubsection{Message Broker (RabbitMQ)}
      Serviciu replicat care preia joburi de procesare de intrevale libere de
      la serviciul calendar și le trimite către un set de workeri proprii.
    \subsubsection{Serviciul Suggestion Worker - replicat}
      Serviciul Suggestion Worker este un modul distribuit care efectuează
      procesare paralelă asupra datelor din baza de date. Preia joburi de la
      broker și procesează în paralel cererile de găsire a unui interval comun
      pentru un grup. Un job conține informații unui grup și a unui interval
      (câteva ore sau zile), iar răspunsul va fi o listă cu intervale de o oră
      disponibile pentru toți membrii grupului în perioada specificată. Un
      worker va prelua disponibilitatea fiecărui membru al grupului în
      intervalul specificat de job și va realiza intersecția acestora pentru a
      găsi posibile locuri libere.
    \subsubsection{Serviciul de locking distribuit (Redis)}
      Gestionează mecanismele de \textit{distributed locking} necesare evitării
      conflictelor  la nivel de interval orar prin algoritmul \textit{RedLock}.
      Mecanismul de locking distribuit este utilizat exclusiv de serviciul
      Calendar pentru a proteja operațiile critice de scriere asupra
      intervalelor orare și a preveni conflictele atunci când mai mulți
      utilizatori sau instanțe ale API-ului încearcă să modifice simultan
      aceleași date.

  \subsection{Rețelele}
    \subsubsection{Rețeaua \texttt{frontend-net}}
      \begin{itemize}
        \item Serviciul Calendar (replicat)
        \item Serviciul de autentificare (Keycloak)
      \end{itemize}

    \subsubsection{Rețeaua \texttt{calendar-net}}
      \begin{itemize}
        \item Serviciul Calendar (replicat)
        \item Message Broker (RabbitMQ)
        \item Baza de date PostgreSQL (instanța pentru calendar)
        \item Redis (locking distribuit)
      \end{itemize}

    \subsubsection{Rețeaua \texttt{profile-net}}
      \begin{itemize}
        \item Serviciul Calendar (replicat)
        \item Serviciul de profil utilizatori
        \item Baza de date PostgreSQL (instanța pentru profiluri)
      \end{itemize}

    \subsubsection{Rețeaua \texttt{worker-net}}
      \begin{itemize}
        \item Message Broker (RabbitMQ)
        \item Suggestion Worker (replicat)
        \item PostgreSQL (instanța pentru calendar)
      \end{itemize}

\section{Diagrama arhitecturii}

\begin{figure}[H]
    \centering
    \includegraphics[width=\textwidth]{img/architecture_diagram.png}
    \caption{Arhitectura platformei de calendar distribuit}
\end{figure}


\end{document}