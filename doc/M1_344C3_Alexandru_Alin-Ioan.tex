\documentclass[a4paper,12pt]{article}

\usepackage[utf8]{inputenc}
\usepackage[romanian]{babel}
\usepackage{amsmath}
\usepackage{amssymb}
\usepackage{graphicx}
\usepackage{hyperref}
\usepackage{geometry}
\usepackage{float}
\usepackage{booktabs}
\usepackage{makecell}
\geometry{margin=1in}

\title{Platformă Distribuită pentru Organizarea Evenimentelor și Sincronizarea Calendarului de Grup}
\author{Alexandru Alin-Ioan \\ \small{Grupa: 344C3}}
\date{\today}

\begin{document}

\maketitle

\tableofcontents

\section{Introducere}

  Această aplicție web propune o soluție pentru un calendar distribuit prin care
  utilizatorii să poată crea, edita și sincroniza evenimente în cadrul unor
  grupuri. Platforma va oferi funcționalitatea de gestionare a disponibilității
  tutror participanților și de găsire automată de intrevale libere pentru
  evenimentele planificate.

\section{Obiectivele aplicației}

  \begin{itemize}
    \item Oferirea unui calendar multi-utilizator cu sincronizare în timp real.
    \item Gestionarea disponibilității și a evenimentelor unui participant sau
    unui grup.
    \item Prevenirea conflictelor de rezervare prin mecanisme de distributed
    locking.
    \item Generarea de propuneri automate de meeting folosind procesare
    asincronă și workeri replicați.
    \item Implementarea unor mecanisme robuste de autentificare, autorizare și
    management al rolurilor folosind Keycloak.
    \item Livrarea proiectului sub forma unui set de servicii Docker.
  \end{itemize}

\section{Descrierea arhitecturii}

  \subsection{Componente principale}
    \begin{enumerate}
      \item \textbf{Serviciul de autetificare (Keycloak)} -- Folosește OAuth2
      și emite token-uri JWT.
      \item \textbf{Serviciul de profil utilizatori} -- Preia informațile oferite
      de Keycloak și creează profiluri locale în baza de date.
      \item \textbf{Baza de date (PostgreSQL)} -- destinată managementului și
      persistenței entităților și setărilor aplicației.
      \item \textbf{Serviciul Calendar} -- expune un API pentru management de
      grupuri, operații CRUD pentru evenimente, verificarea disponibilității.
      \item \textbf{Serviciul Suggestion Worker (replicat)} -- preia joburi de
      la broker și procesează în paralel cererile de găsire a unui interval
      comun pentru un grup.
      \item \textbf{Message Broker (RabbitMQ)} -- preia cererile de
      procesare a unui interval de la serviciul calendar și le trimite către
      workeri.
      \item \textbf{Serviciul de locking distribuit (Redis)} -- gestionează 
      mecanismele de \textit{distributed locking} necesare evitării conflictelor 
      la nivel de interval orar prin algoritmul \textit{RedLock}.
    \end{enumerate}

  \subsection{Rețele și comunicare între microservicii}

\section{Modulele aplicației}

  \begin{enumerate}
    \item \textbf{Autentificare} -- funcție de autorizare utilizând o tehnologie
    de Single Sign-On (SSO).
    \item \textbf{Profil utilizator + roluri} -- Keycloak pentru managementul
    utilizatorilor, rolurilor și token-urilor.
    \item \textbf{Baza de date} -- gestionată prin ORM.
    \item \textbf{Conflict resolution + Distributed Locking} -- gestionează
    situațiile concurente în care mai mulți utilizatori creează sau modifică
    evenimente în același interval temporal.
    \item \textbf{Suggestion Worker replicat} -- calculează automat intervalele comune
    libere pentru un grup. Joburile sunt transmise de către serviciul Calendar
    către broker, care mai apoi le redistribuie către workeri.
  \end{enumerate}

\section{Diagrama arhitecturii}

\begin{figure}[H]
    \centering
    \includegraphics[width=\textwidth]{img/architecture_diagram.png}
    \caption{Arhitectura platformei de calendar distribuit}
\end{figure}


\end{document}